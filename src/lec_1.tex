\documentclass[../main.tex]{subfiles}

\begin{document}
        \section{Vector-Valued Functions}
        \begin{definition}{Vector-Valued Functions}{}
                A vector valued function is a function \(\vec{r}:I \to \mathbb{R}^n\), where \(I \subseteq \mathbb{R}\). A vector valued function in \(\mathbb{R}^3\) may be written in the form
                \[
                \vec{r}(t)=x(t)\i + y(t)\j + z(t)\k
                \]
        \end{definition}
        \begin{definition}{Continuity}{}
                A vector valued function \(\vec{r}(t)=x(t)\i + y(t)\j + z(t)\k
\) is \emph{continuous} on \(I\) if and only if \(x(t)\), \(y(t)\), and \(z(t)\) are all continuous on \(I\).
        \end{definition}
        \begin{definition}{Velocity \& Acceleration}{}
                Given some vector valued function \(\vec{r}(t)\), there exists a corresponding \emph{velocity} \(\vec{v}(t)\) equal to \(\vec{r}'(t)\) and a corresponding \emph{acceleration} \(\vec{a}(t)\) equal to \(\vec{r}''(t)\), or equivalently \(\vec{v}'(t)\). The magnitude of \(\vec{v}(t)\) is a scalar quantity sometimes called the \emph{speed} of \(\vec{r}\).
        \end{definition}
        A stright line is described by the parameterization
        \begin{equation*}
                \vec{r}(t)=\vec{r}_0 + t\vec{v}
        \end{equation*}
        but may also be equivalently parameterized like
        \begin{equation*}
                \vec{r}(t)=\vec{r}_0 + 2t\vec{v}
        \end{equation*}
        which describes the same curve, but has a different velocity. Generally, the parameterization of a curve is not unique.
        
        A helix is descibed by the parameterization
        \begin{equation*}
                \vec{r}(t) = \cos{t}\i + \sin{t}\j + t\k
        \end{equation*}
        which corresponds to uniform circular motion in the x-y plane coupled with downward motion in the z direction with constant velocty, resulting in a curve with a helical shape. \(z(t)\) may be modified to change the ``compression" of the helix. For instance,
        \begin{equation*}                        
                \vec{r}(t) = \cos{t}\i + \sin{t}\j + t^3\k
        \end{equation*}
        is much less compressed at large values of \(z\), and highly compressed near \(0\).
        \begin{example}{}{}
        Parameterize the curve of intersection of \(x^2+y^2=9\) with \(z = x + y\).
        \tcblower
        In the x-y plane, the path is circular, so we have
        \begin{align*}
                x &= 3\cos{t}\\
                y &= 3\sin{t}
        \end{align*}
        and due to the fact that the curve lies on the surface \(z = x + y\), it must be that
        \[z = 3\cos{t} + 3\sin{t}\]
        which is a parameterization on the interval \([0,2\pi]\).
        \end{example}
        \begin{example}{}{}
        Parameterize the intersection of \(z = x^2 + y^2\) and \(x = 2y\).
        \tcblower
        The intersection of these curves is a parabola, so we may simply set
        \[y=t\]
        which then yields equations for \(x\) and \(z\)
        \begin{align*}
                x &= 2t\\
                z &= 5t^2
        \end{align*}
        \end{example}
        \begin{definition}{Differentiation Rules}{}
                The differentiation of vector valued functions proceeds by differentiating each of the components, so the normal rules of differentiation apply, namely
                \[(\vec{u}+\vec{v})' = \vec{u}' + \vec{v}'\]
                \[(\vec{u}\cdot\vec{v})' = \vec{u}'\cdot\vec{v}+\vec{u}\cdot\vec{v}'\]
                and similarly for the cross product
                \[(\vec{u}\times\vec{v})' = \vec{u}'\times\vec{v}+\vec{u}\times\vec{v}'\]
        \end{definition}
\end{document}
