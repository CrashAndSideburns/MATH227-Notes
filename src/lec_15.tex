\documentclass[../main.tex]{subfiles}

\begin{document}
    \section{Parametric Surfaces}
    We consider parametric surfaces, which are higher-dimensional analogues of parametric curves. A parametric surface is parameterized by a function \(\vec{r}:D\to\mathbb{R}^3\), where \(\vec{r}(u,v)\) is continuous and \(D\subseteq\mathbb{R}^2\) is a domain. As with curves, a given surface may be parameterized in many ways.

    The graph of a function \(z=f(x,y)\) can be thought of as a parametric surface parameterized by 
    \[
    \vec{r}(u,v)=u\i+v\j+f(u,v)\k
    \]

    Spheres are also common parametric surfaces. In spherical coordinates, we may parameterize the sphere in terms of \(\theta\) and \(\phi\) as
    \[
    \vec{r}(\theta,\phi)=a\cos{\theta}\sin{\phi}\i+a\sin{\theta}\cos{\phi}\j+a\cos{\phi}\k
    \]
    where \(\theta\in[0,2\pi]\) and \(\phi\in[0,\pi]\).

    There is also an analogue to the helical curve, which resembles a spiral stairway, and is parameterized by
    \[
    \vec{r}(u,v)=u\cos{v}\i+u\sin{v}\j+v\k
    \]
    where \(u\in[0,1]\) and \(v\in[0,2\pi]\).

    We generally assume, when working with surfaces, that \(\vec{r}\) and its derivatives are continuous.
    
    \begin{definition}{Surface Closure}{}
        A surface is \emph{closed} if it bounds a closed subset of \(\mathbb{R}^3\).
    \end{definition}

    As with parametric curves, we wish to find equations for the normal vectors of surfaces, as well as their tangent planes. It is also necessary to find area elements of surfaces, enabling us to integrate over them.

    We wish to find the normal vector to a parametric surface parameterized by \(\vec{r}\) at some point \(P=\vec{r}(u_0,v_0)\). We can construct a pair of parametric curves parameterized by \(\vec{r}(u,v_0)\) and \(\vec{r}(u_0,v)\), both of which pass through \(P\). The normal vector of the surface at \(P\) must then be perpendicular to both curves, or equivalently to their tangent vectors at \(P\). This yields the equation
    \[
    \vec{n}=\vec{r}_u\times\vec{r}_v
    \]

    This normal vector is not independent of parameterization, but it does always point in the direction perpendicular to the surface.

    We also wish to find area elements of our surface by continually subdividing the surface along lines with constant \(u\) and constant \(v\). As we continually take smaller and smaller subdivisions, the area element approximates a parallelogram. The vectors spanning this parallelogram are \(\vec{r}_u\,du\) and \(\vec{r}_v\,dv\), so the area element is given by
    \[
    dS=|\vec{r}_u\,du\times\vec{r}_v\,dv|=|\vec{r}_u\times\vec{r}_v|\,du\,dv=|\vec{n}|\,du\,dv
    \]

    Using this, we can integrate functions along surfaces. If we wish to compute the integral of some function \(f\) along some surface \(\mathcal{S}\), we have
    \[
    \iint_\mathcal{S}f(x,y,z)\,dS=\iint_Df(\vec{r}(u,v))|\vec{n}|\,du\,dv
    \]

    One simple application of surface integrals is computing the areas of surfaces. The area of a surface \(\mathcal{S}\) is given by
    \[
    \iint_\mathcal{S}\,dS
    \]

    We can also compute the masses of surfaces with variable densities. The mass of a surface \(\mathcal{S}\) with variable density \(\rho\) is given by
    \[
    \iint_\mathcal{S}\rho\,dS
    \]

    It is also possible to compute the average value of functions along surfaces. The average value of \(f\) on \(\mathcal{S}\) is given by
    \[
    \frac{\iint_\mathcal{S}f\,dS}{\iint_\mathcal{S}\,dS}
    \]
    
    \begin{example}{}{}
        Find the area of the part of the plane \(2x-y+z=10\) which lies above the disk \((x-3)^2+y^2\geq1\).
        \tcblower
        We begin by parameterizing the plane. The plane may be treated as a graph, and is parameterized by
        \[
        \vec{r}(x,y)=x\i+y\j+(10-2x+y)\k
        \]

        Which has \(\vec{r}_x=\i-2\k\) and \(\vec{r}_y=\j+\k\), so we compute
        \[
        \vec{n}=\vec{r}_x\times\vec{r}_y=2\i-\j+\k
        \]
        and
        \[
        |\vec{n}|=\sqrt{6}
        \]
        so the area of the ellipse is
        \[
        \iint_D\sqrt{6}\,dx\,dy
        \]
        where \(D\) is the disk. Normally, we would have to compute a double integral, but because \(|\vec{n}|\) is constant and \(D\) is simply a circle, we have
        \[
        \iint_D\sqrt{6}\,dx\,dy=\sqrt{6}\pi
        \]
    \end{example}

        The problem of finding the area of a graph may be solved more generally. Let \(\mathcal{S}\) be the graph of a function \(z=f(x,y)\), where \((x,y)\in D\). We wish to find the area of \(\mathcal{S}\). Because \(\mathcal{S}\) may be parameterized by
        \[
        \vec{r}(x,y)=x\i+y\j+f(x,y)\k
        \]
        which immediately yields \(\vec{r}_x=\i+f_x\k\) and \(\vec{r}_y=\j+f_y\k\), and consequently
        \[
        \vec{n}=-f_x\i-f_y\j+\k
        \]
        which yields
        \[
        |\vec{n}|=\sqrt{f_x^2+f_y^2+1}
        \]

        As such, the area of the graph \(z=f(x,y)\) is simply given by
        \[
        \iint_D\sqrt{f_x^2+f_y^2+1}\,dx\,dy
        \]

        \begin{example}{}{}
            Find the area of the part of \(z=x^2+y^2\) below \(z=9\).
            \tcblower
            First, we compute
            \[
            |\vec{n}|=\sqrt{z_x^2+z_y^2+1}
            \]
            meaning that the area is
            \[
            \iint_D\sqrt{4x^2+4y^2+1}\,dx\,dy=\frac{\pi}{6}(37\sqrt{37}-1)
            \]
            where the integral is computed by conversion to polar coordinates, and \(D=\{(x,y):x^2+y^2<9\}\).
        \end{example}
\end{document}
