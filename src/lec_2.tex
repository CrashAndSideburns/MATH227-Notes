\documentclass[../main.tex]{subfiles}

\begin{document}
        \section{Curves \& Arc Length}
        \begin{definition}{Smoothness}{}
                A vector valued function \(\vec{r}(t)\) is \emph{smooth} on an interval \((a,b)\) if \(\vec{r}'(t)\) exists \emph{and} \(\vec{r}'(t)\neq\vec{0}\) on the interval.
        \end{definition}
        In the above definition, the case where \(\vec{r}'(t)=\vec{0}\) corresponds to coming to a full stop, at which point motion may resume in any direction while maintaining the existence of the derivative, so a stronger condition is necessary to eliminate these cases.
        \begin{definition}{Closure}{}
                A vector valued function \(\vec{r}(t)\) is \emph{closed} on an interval \((a,b)\) if \(\vec{r}(a)=\vec{r}(b)\).
        \end{definition}
        \begin{definition}{Simplicity}{}
                A vector valued function \(\vec{r}(t)\) is \emph{simple} on an interval \((a,b)\) if it has no self-intersections on the interval, except possibly where \(\vec{r}(a)=\vec{r}(b)\).
        \end{definition}
        Given a curve described by some function \(\vec{r}(t)\) on some interval \((a,b)\), we may wish to find its arc length. We assume that \(\vec{r}\) is continuous, smooth, and 1-to-1. This may be done by approximating the curve as a sequence of line segments, and taking the limit as their length goes to \(0\). We consider some set of points \(\{t_0,\ldots,t_n\}\), where \(a=t_0 < t_1 < \ldots < t_n=b\), which yield the corresponding line segments \(\{\vec{r}(t_1)-\vec{r}(t_0),\ldots,\vec{r}(t_n)-\vec{r}(t_{n-1})\}\) whose lengths may be summed up to obtain an approximation.
        \begin{definition}{Rectifiability}{}
                A curve is rectifiable if there exists some \(k>0\) such that the length of an approximation of the curve in terms of line segments is less than \(k\) for any number of line segments. In other words, if the length of the approximation approaches a limit.
        \end{definition}
        If a curve is rectifiable with some \(k\), then the smallest such \(k\) is the length of the curve. Formulaically,
        \[
        L = \int_a^b \left| \vec{r}'(t) \right|\, dt = \int_a^b v(t)\,dt
        \]
        which may be proven by noting that the length of the line segment between \(\vec{r}(t_i)\) and \(\vec{r}(t_{i-1})\) is \(|\vec{r}(t_i)-\vec{r}(t_{i-1})|\), which approaches \(|\vec{v}'(t_i)|\) as \(t_i\) and \(t_{i-1}\) become very close.
        \begin{example}{}{}
                Find the length of the helix described by \(\vec{r}(t)=a\cos{t}\i+a\sin{t}\j+bt\) on the interval \(0\leq t \leq T\).
                \tcblower
                \(\vec{r}'(t)=-a\sin{t}\i+a\cos{t}\j+b\k\), so \(v(t)=\sqrt{a^2\sin^2{t}+a^2\cos^2{t}+b^2}=\sqrt{a^2+b^2}\) and
                \[
                L = \int_0^T \sqrt{a^2+b^2}\,dt = \left[\sqrt{a^2+b^2}t\right]_0^T = T\sqrt{a^2+b^2}
                \]
        \end{example}
        \begin{example}{}{}
                Find the length of the curve described by \(\vec{r}(t)=2t\i+t^2\j+\frac{1}{3}t^3\k\) on the interval \(1\leq t \leq 2\).
                \tcblower
                \(\vec{r}'(t)=2\i+2t\j+t^2\k\), so \(v(t)=\sqrt{4+4t^2+t^4}=\sqrt{(2+t^2)^2}=2+t^2\), and the length of the curve is
                \[
                L = \int_1^2 2+t^2\,dt = \left[2t+\frac{t^3}{3}\right]_1^2 = \frac{13}{3}
                \]
        \end{example}
        Let's say that \(\vec{r}:[a,b]\to\mathbb{R}^d\) is a smooth parametric curve, and \(s(t)\) is the length of the curve from \(\vec{r}(a)\) to \(\vec{r}(t)\). We may then use \(s\) to parametrize the curve instead of \(t\), which yields a parameterization with a constant speed of \(1\).

        Note that
        \[
        s(t) = \int_a^t v(u)\,du
        \]
        so
        \[
        \frac{ds}{dt}=v(t)
        \]
        by FTC.
        \begin{theorem}{}{}
                Prove that an arc length parametrized curve has constant speed \(1\).
                \tcblower
                Note that
                \[
                \left|\frac{d\vec{r}}{dt}\right|=\left|\frac{d\vec{r}}{dt}\frac{dt}{ds}\right|
                \]
                but by the definition of arc length parameterization we have \(\frac{dt}{ds}=\frac{1}{v(t)}\), which is always defined because \(\vec{r}\) is smooth. In that case,
                \[
                \frac{d\vec{r}}{dt}=|v(t)|\frac{1}{|v(t)|}=1
                \]
        \end{theorem}
        \begin{example}{}{}
                Find the arc length parameterization of the helix described by \(\vec{r}(t)=a\cos{t}\i+a\sin{t}\j+bt\) on the interval \(0\leq t \leq T\).
                \tcblower
                \(\vec{r}(t)=a\cos{t}\i+a\sin{t}\j+b\k\), so \(v(t)=\sqrt{a^2+b^2}\) and
                \[
                s(t) = \int_0^t \sqrt{a^2+b^2}\,du = t\sqrt{a^2+b^2}
                \]
                which we may use this to reparameterize the curve in the form
                \[
                        \vec{r}=a\cos\left(\frac{s}{\sqrt{a^2+b^2}}\right)\i + a\sin\left(\frac{s}{\sqrt{a^2+b^2}}\right)\j + b\frac{s}{\sqrt{a^2+b^2}}\k
                \]
        \end{example}
        \begin{example}{}{}
                Find the arc length parameterization of the curve described by \(\vec{r}(t)=2t\i+t^2\j+\frac{t^3}{3}\k\) on the interval \(0 \leq t \leq T\).
                \tcblower
                \(\vec{r}(t)=2t\i+t^2\j+\frac{t^3}{3}\k\), so \(v(t)=t^2+2\) and
                \[
                s(t) = \int_0^tu^2+2\,du = \frac{t^3}{3}+2t
                \]
                which may be solved for \(t\) to find a parameterization. The algebra involved is tedious, and thus elided.
        \end{example}
\end{document}
